\documentclass[landscape,final,a0paper,fontscale=0.27065]{baposter}

\usepackage{calc}
\usepackage{graphicx}
\usepackage{amsmath}
\usepackage{amssymb}
\usepackage{relsize}
\usepackage{multirow}
\usepackage{rotating}
\usepackage{bm}
\usepackage{enumitem}
\usepackage{url}
\usepackage{booktabs}

\usepackage{graphicx}
\usepackage{multicol}

%\usepackage{times}
%\usepackage{helvet}
%\usepackage{bookman}
\usepackage{palatino}

\newcommand{\captionfont}{\footnotesize}

\graphicspath{{images/}{../images/}}
\usetikzlibrary{calc}

\newcommand{\Matrix}[1]{\begin{bmatrix} #1 \end{bmatrix}}
\newcommand{\Vector}[1]{\begin{pmatrix} #1 \end{pmatrix}}

\newcommand*{\norm}[1]{\mathopen\| #1 \mathclose\|}% use instead of $\|x\|$
\newcommand*{\abs}[1]{\mathopen| #1 \mathclose|}% use instead of $\|x\|$
\newcommand*{\normLR}[1]{\left\| #1 \right\|}% use instead of $\|x\|$

\newcommand*{\SET}[1]  {\ensuremath{\mathcal{#1}}}
\newcommand*{\FUN}[1]  {\ensuremath{\mathcal{#1}}}
\newcommand*{\MAT}[1]  {\ensuremath{\boldsymbol{#1}}}
\newcommand*{\VEC}[1]  {\ensuremath{\boldsymbol{#1}}}
\newcommand*{\CONST}[1]{\ensuremath{\mathit{#1}}}

\DeclareMathOperator*{\argmax}{arg\,max}
\DeclareMathOperator*{\diag}{diag}
\DeclareMathOperator*{\argmin}{arg\,min}
\DeclareMathOperator*{\vectorize}{vec}
\DeclareMathOperator*{\reshape}{reshape}

%\font\dsfnt=dsrom12

\newcommand{\SNN}{\ensuremath{\mathbb N}}
\newcommand{\SRR}{\ensuremath{\mathbb R}}
\newcommand{\SZZ}{\ensuremath{\mathbb Z}}
%-----------------------------------------------------------------------------
% Matrices of the shape model
\renewcommand{\a}{\VEC\alpha}
\renewcommand{\v}{\VEC v}
\renewcommand{\l}{\VEC l}
\newcommand*{\m}{\VEC{\mu}}
\newcommand*{\M}{\MAT{M}}
\renewcommand*{\P}{\MAT{\Pi}}

%\newcommand{\J}{\SET J}
\newcommand{\J}{\SET{P}}
\newcommand{\Active}{\mathcal{A}}
\newcommand{\Selection}{\mathbf{S}}
\newcommand{\AllSelections}{\mathfrak{S}}
\newcommand{\Params}{\VEC\Theta}

%%%%%%%%%%%%%%%%%%%%%%%%%%%%%%%%%%%%%%%%%%%%%%%%%%%%%%%%%%%%%%%%%%%%%%%%%%%%%%%%
%%%% Some math symbols used in the text
%%%%%%%%%%%%%%%%%%%%%%%%%%%%%%%%%%%%%%%%%%%%%%%%%%%%%%%%%%%%%%%%%%%%%%%%%%%%%%%%

%%%%%%%%%%%%%%%%%%%%%%%%%%%%%%%%%%%%%%%%%%%%%%%%%%%%%%%%%%%%%%%%%%%%%%%%%%%%%%%%
% Multicol Settings
%%%%%%%%%%%%%%%%%%%%%%%%%%%%%%%%%%%%%%%%%%%%%%%%%%%%%%%%%%%%%%%%%%%%%%%%%%%%%%%%
\setlength{\columnsep}{1.5em}
\setlength{\columnseprule}{0mm}

%%%%%%%%%%%%%%%%%%%%%%%%%%%%%%%%%%%%%%%%%%%%%%%%%%%%%%%%%%%%%%%%%%%%%%%%%%%%%%%%
% Save space in lists. Use this after the opening of the list
%%%%%%%%%%%%%%%%%%%%%%%%%%%%%%%%%%%%%%%%%%%%%%%%%%%%%%%%%%%%%%%%%%%%%%%%%%%%%%%%
\newcommand{\compresslist}{%
\setlength{\itemsep}{1pt}%
\setlength{\parskip}{0pt}%
\setlength{\parsep}{0pt}%
}

%%%%%%%%%%%%%%%%%%%%%%%%%%%%%%%%%%%%%%%%%%%%%%%%%%%%%%%%%%%%%%%%%%%%%%%%%%%%%%
%%% Begin of Document
%%%%%%%%%%%%%%%%%%%%%%%%%%%%%%%%%%%%%%%%%%%%%%%%%%%%%%%%%%%%%%%%%%%%%%%%%%%%%%

\begin{document}

 %%%%%%%%%%%%%%%%%%%%%%%%%%%%%%%%%%%%%%%%%%%%%%%%%%%%%%%%%%%%%%%%%%%%%%%%%%%%%%
 %%% Here starts the poster
 %%%---------------------------------------------------------------------------
 %%% Format it to your taste with the options
 %%%%%%%%%%%%%%%%%%%%%%%%%%%%%%%%%%%%%%%%%%%%%%%%%%%%%%%%%%%%%%%%%%%%%%%%%%%%%%
 % Define some colors
 
% \definecolor{lightorange}{rgb}{0.9,0.4,0}
% \definecolor{lightestorange}{rgb}{1,0.9,0.6}
% \definecolor{darkorange}{rgb}{0.2,0.1,0}
%\definecolor{lightorange}{rgb}{0.9,0.3,0}
%\definecolor{lightestorange}{rgb}{1,0.8,0.5}
%\definecolor{darkorange}{rgb}{0.2,0.1,0}
%\definecolor{darkorange}{rgb}{0.8,0.1,0}
%\definecolor{darkorange}{cmyk}{0,0.40,0.86,0.33}
\colorlet{darkorange}{orange!30!black}
\colorlet{lightorange}{white}
\colorlet{lightestorange}{orange!65!white}
 
 \hyphenation{resolution occlusions}
% %%
\begin{poster}%
{
  grid=false,
%  % Color style
  bgColorOne=lightestorange!50!white,
  bgColorTwo=lightestorange,
  borderColor=lightorange,
  headerColorOne=darkorange,
  headerColorTwo=lightorange,
  headerFontColor=white,
  boxColorOne=lightestorange,
  boxColorTwo=lightorange,
%  % Format of textbox
  textborder=faded,
%  % Format of text header
  headerborder=open,
  headerheight=0.13\textheight,
%  textfont=\sc,% An example of changing the text font
  headershape=rounded,
  headershade=plain,
  headerfont=\Large\bf\textsc, %Sans Serif
  textfont={\setlength{\parindent}{0.5em}},
  boxshade=none,
  background=shadetb,
%  background=plain,
  linewidth=4pt
  }{\includegraphics[height=7em]{images/search_tree_ex1-crop.pdf}}{
  \bf\textsc{Optimal Landmark Detection using\\[0.2em]Shape Models and Branch and Bound}\vspace{0.5em}
  }{
    \textsc{\{ Brian.Amberg and Thomas.Vetter \}@unibas.ch}}{
      \includegraphics[height=9.0em]{images/logo}
    }
%%%%%%%%%%%%%%%%%%%%%%%%%%%%%%%%%%%%%%%%%%%%%%%%%%%%%%%%%%%%%%%%%%%%%%%%%%%%%%
%%% Now define the boxes that make up the poster
%%%---------------------------------------------------------------------------
%%% Each box has a name and can be placed absolutely or relatively.
%%% The only inconvenience is that you can only specify a relative position 
%%% towards an already declared box. So if you have a box attached to the 
%%% bottom, one to the top and a third one which should be in between, you 
%%% have to specify the top and bottom boxes before you specify the middle 
%%% box.
%%%%%%%%%%%%%%%%%%%%%%%%%%%%%%%%%%%%%%%%%%%%%%%%%%%%%%%%%%%%%%%%%%%%%%%%%%%%%%
%
% A coloured circle useful as a bullet with an adjustably strong filling
\newcommand{\colouredcircle}{%
  \tikz{\useasboundingbox (-0.2em,-0.32em) rectangle(0.2em,0.32em); \draw[draw=black,fill=lightblue,line width=0.03em] (0,0) circle(0.18em);}}

  %%%%%%%%%%%%%%%%%%%%%%%%%%%%%%%%%%%%%%%%%%%%%%%%%%%%%%%%%%%%%%%%%%%%%%%%%%%%%%
  \begin{posterbox}[name=problem,column=0,row=0]{Problem}
    %%%%%%%%%%%%%%%%%%%%%%%%%%%%%%%%%%%%%%%%%%%%%%%%%%%%%%%%%%%%%%%%%%%%%%%%%%%%%%
    Given an image and a (2D or 3D) shape model, detect a large set of fiducials defined in the shape model.

    Searching for the fiducials independently fails, because
    \begin{itemize}
    \setlength{\itemsep}{0em}
    \item locally, many patches look the same
    \item some fiducials can be occluded 
    \end{itemize}
    
    Globally fitting the shape model fails, because
    \begin{itemize}
    \setlength{\itemsep}{0em}
    \item the search problem is too large, 
    \item and the cost surface too complex
    \end{itemize}
  \end{posterbox}
 

 %%%%%%%%%%%%%%%%%%%%%%%%%%%%%%%%%%%%%%%%%%%%%%%%%%%%%%%%%%%%%%%%%%%%%%%%%%%%%%
 \begin{posterbox}[name=results,column=1,span=2,row=0,textborder=none]{}
   %%%%%%%%%%%%%%%%%%%%%%%%%%%%%%%%%%%%%%%%%%%%%%%%%%%%%%%%%%%%%%%%%%%%%%%%%%%%%%
   {
 \smaller\centering
 \begin{tabular}{@{}rccccccc@{}}
 \begin{sideways}\makebox[0pt][c]{Success}\end{sideways} &
 \parbox[c]{0.11\linewidth}{\includegraphics[width=\linewidth]{images/l_fa_success_1.pdf}} &
 \parbox[c]{0.11\linewidth}{\includegraphics[width=\linewidth]{images/l_fb_success_1.pdf}} &
 \parbox[c]{0.11\linewidth}{\includegraphics[width=\linewidth]{images/l_ql_success_1.pdf}} &
 \parbox[c]{0.11\linewidth}{\includegraphics[width=\linewidth]{images/l_qr_success_1.pdf}} &
 \parbox[c]{0.11\linewidth}{\includegraphics[width=\linewidth]{images/l_hl_success_1.pdf}} &
 \parbox[c]{0.11\linewidth}{\includegraphics[width=\linewidth]{images/l_hr_success_1.pdf}} &
 \parbox[c]{0.11\linewidth}{\includegraphics[width=\linewidth]{images/l_rc_success_1.pdf}} \\
 &
 \parbox[c]{0.11\linewidth}{\includegraphics[width=\linewidth]{images/l_fa_success_2.pdf}} &
 \parbox[c]{0.11\linewidth}{\includegraphics[width=\linewidth]{images/l_fb_success_2.pdf}} &
 \parbox[c]{0.11\linewidth}{\includegraphics[width=\linewidth]{images/l_ql_success_2.pdf}} &
 \parbox[c]{0.11\linewidth}{\includegraphics[width=\linewidth]{images/l_qr_success_2.pdf}} &
 \parbox[c]{0.11\linewidth}{\includegraphics[width=\linewidth]{images/l_hl_success_2.pdf}} &
 \parbox[c]{0.11\linewidth}{\includegraphics[width=\linewidth]{images/l_hr_success_2.pdf}} &
 \parbox[c]{0.11\linewidth}{\includegraphics[width=\linewidth]{images/l_rc_success_2.pdf}} \\
 \midrule
 \begin{sideways}\makebox[0pt][c]{Failure}\end{sideways} &
 \parbox[c]{0.11\linewidth}{\includegraphics[width=\linewidth]{images/l_fa_fail.pdf}} &
 \parbox[c]{0.11\linewidth}{\includegraphics[width=\linewidth]{images/l_fb_fail.pdf}} &
 \parbox[c]{0.11\linewidth}{\includegraphics[width=\linewidth]{images/l_ql_fail.pdf}} &
 \parbox[c]{0.11\linewidth}{\includegraphics[width=\linewidth]{images/l_qr_fail.pdf}} &
 \parbox[c]{0.11\linewidth}{\includegraphics[width=\linewidth]{images/l_hl_fail.pdf}} &
 \parbox[c]{0.11\linewidth}{\includegraphics[width=\linewidth]{images/l_hr_fail.pdf}} &
 \parbox[c]{0.11\linewidth}{\includegraphics[width=\linewidth]{images/l_rc_fail.pdf}} 
 \end{tabular}
   }\\[-1em]
       \begin{multicols}{2}
 Some randomly chosen images from the color feret database for each
 pose, and the detected landmark positions. The first two rows are success
 cases, the last row shows a failure case. 
       \end{multicols}
 \end{posterbox}
%%%%%%%%%%%%%%%%%%%%%%%%%%%%%%%%%%%%%%%%%%%%%%%%%%%%%%%%%%%%%%%%%%%%%%%%%%%%%%
  \begin{posterbox}[name=references,column=0,above=bottom]{References}
%%%%%%%%%%%%%%%%%%%%%%%%%%%%%%%%%%%%%%%%%%%%%%%%%%%%%%%%%%%%%%%%%%%%%%%%%%%%%%
    \smaller
    \bibliographystyle{ieee}
    \renewcommand{\section}[2]{\vskip 0.05em}
      \begin{thebibliography}{1}\itemsep=-0.01em
      \setlength{\baselineskip}{0.4em}
      \bibitem{amberg11:bnb}
        B.~Amberg, T. Vetter.
        \newblock {O}ptimal {L}andmark {D}etection using {S}hape {M}odels and {B}ranch and {B}ound
        \newblock In {\em ICCV '11}
      \end{thebibliography}
   \vspace{0.3em}
  \end{posterbox}
%%%%%%%%%%%%%%%%%%%%%%%%%%%%%%%%%%%%%%%%%%%%%%%%%%%%%%%%%%%%%%%%%%%%%%%%%%%%%%
\begin{posterbox}[name=contribution,column=0,below=problem,above=references]{Contributions}
  %%%%%%%%%%%%%%%%%%%%%%%%%%%%%%%%%%%%%%%%%%%%%%%%%%%%%%%%%%%%%%%%%%%%%%%%%%%%%%
  We integrate local detection and global shape fitting by
  \begin{itemize}
    \setlength{\itemsep}{0em}
  \item Detecting fiducials independently, but noisily (many false positives, few false negatives)
  \item Finding the subset of detections and shape model parameters which best fit together
  \end{itemize}

  This has been addressed with RANSAC before, but our proposal
  \begin{itemize}
    \setlength{\itemsep}{0em}
  \item Finds the optimal solution (for some shape models with guarantees)
  \item Is very efficient
  \end{itemize}

  Our method can be 
  \begin{itemize}
    \setlength{\itemsep}{0em}
  \item combined with any local detector
  \item used for a wide range of shape models
  \end{itemize}
\end{posterbox}
%%%%%%%%%%%%%%%%%%%%%%%%%%%%%%%%%%%%%%%%%%%%%%%%%%%%%%%%%%%%%%%%%%%%%%%%%%%%%%
  \begin{posterbox}[name=source,column=3,above=bottom]{Source Code}
%%%%%%%%%%%%%%%%%%%%%%%%%%%%%%%%%%%%%%%%%%%%%%%%%%%%%%%%%%%%%%%%%%%%%%%%%%%%%%
  \noindent
  \begin{minipage}{\linewidth}
  \begin{minipage}{0.75\linewidth}
    \indent{}The source code is available at \\
    \url{http://www.cs.unibas.ch/personen/amberg_brian/bnb/}
  \end{minipage}\hfill%
  \begin{minipage}{0.23\linewidth}
  \hfill\includegraphics[width=\linewidth]{chart}
  \end{minipage}
  \end{minipage}
  \end{posterbox}
%%%%%%%%%%%%%%%%%%%%%%%%%%%%%%%%%%%%%%%%%%%%%%%%%%%%%%%%%%%%%%%%%%%%%%%%%%%%%%
  \begin{posterbox}[name=solution1,column=1,above=bottom]{Solution}
%%%%%%%%%%%%%%%%%%%%%%%%%%%%%%%%%%%%%%%%%%%%%%%%%%%%%%%%%%%%%%%%%%%%%%%%%%%%%%
This mixed continuous/discrete problem is solved with Branch and Bound, by
repeatedly splitting the set of solutions, and determining a lower bound on the
cost of each subset.
  \end{posterbox}
%%%%%%%%%%%%%%%%%%%%%%%%%%%%%%%%%%%%%%%%%%%%%%%%%%%%%%%%%%%%%%%%%%%%%%%%%%%%%%
  \begin{posterbox}[name=formulation,column=1,below=results,above=solution1]{Formulation}
%%%%%%%%%%%%%%%%%%%%%%%%%%%%%%%%%%%%%%%%%%%%%%%%%%%%%%%%%%%%%%%%%%%%%%%%%%%%%%
Given a shape model 
\begin{align}
  \VEC M(\Params) = \{\VEC m_i(\Params) \in \SRR^2 \mid i\in 1\dots N\}
\end{align}
and landmark candidates $\l_i^j \in \SRR^2$
we find model parameters $\Params$ and a selection of landmarks $j_i$ 
minimizing
\begin{align}
  F(\Params, j_1, \dots, j_N) = \sum \rho\left( \normLR{ \VEC m_i(\Params) - \l_i^{j_i} } \right).
\end{align}
Here $\rho: \SRR\to\SRR$ is a robust cost function.
  \end{posterbox}
%%%%%%%%%%%%%%%%%%%%%%%%%%%%%%%%%%%%%%%%%%%%%%%%%%%%%%%%%%%%%%%%%%%%%%%%%%%%%%
  \begin{posterbox}[name=representation,column=2,above=bottom,below=results]{Representation}
%%%%%%%%%%%%%%%%%%%%%%%%%%%%%%%%%%%%%%%%%%%%%%%%%%%%%%%%%%%%%%%%%%%%%%%%%%%%%%
  A subset of solutions is described by $N$ convex polygons. A subset consists of all selections
  \begin{align}
    \{ j_i \mid \l_i^{j_i} \in \text{polygon $i$}\}\quad.
  \end{align}

  A lower bound on the quality of fit towards any selection in this subset is
  given by a fit to the enclosing convex polygon.
  \vspace{1.0em}
  \begin{center}
    \includegraphics[width=\linewidth]{images/representation.pdf}
  \end{center}
  \end{posterbox}
%%%%%%%%%%%%%%%%%%%%%%%%%%%%%%%%%%%%%%%%%%%%%%%%%%%%%%%%%%%%%%%%%%%%%%%%%%%%%%
\begin{posterbox}[name=strategy,column=3,row=0]{Splitting Strategy}
  %%%%%%%%%%%%%%%%%%%%%%%%%%%%%%%%%%%%%%%%%%%%%%%%%%%%%%%%%%%%%%%%%%%%%%%%%%%%%%
  {\smaller\centering{Runtime as a function of the splitting strategy}\\[-0.5em]
    \includegraphics[width=0.82\linewidth]{images/typical_random_splitting_strategies.pdf}}\\
    Different splitting strategies result in vastly different performance.
      Note that  `split into equal sized problems' is one of the worst strategies for
      branch and bound.
\end{posterbox}
%%%%%%%%%%%%%%%%%%%%%%%%%%%%%%%%%%%%%%%%%%%%%%%%%%%%%%%%%%%%%%%%%%%%%%%%%%%%%%
\begin{posterbox}[name=scaling,column=3,below=strategy,above=source]{Scaling Behaviour}%
%%%%%%%%%%%%%%%%%%%%%%%%%%%%%%%%%%%%%%%%%%%%%%%%%%%%%%%%%%%%%%%%%%%%%%%%%%%%%%
  \smaller%
  \centering{Runtime as a function of the number of false positives}\\[0em]%
  \centering{{\includegraphics[width=0.82\linewidth]{images/typical_random_no_noise-crop.pdf}}}\\[0em]%
  \centering{Runtime as a function of detection accuracy}\\[0em]%
  \centering{{\includegraphics[width=0.82\linewidth]{images/typical_random_add_noise-crop.pdf}}}\\[0em]%
\end{posterbox}
% %%%%%%%%%%%%%%%%%%%%%%%%%%%%%%%%%%%%%%%%%%%%%%%%%%%%%%%%%%%%%%%%%%%%%%%%%%%%%%
% \begin{posterbox}[name=solution,column=2,row=0,below=results,above=bottom]{Solution}
%   %%%%%%%%%%%%%%%%%%%%%%%%%%%%%%%%%%%%%%%%%%%%%%%%%%%%%%%%%%%%%%%%%%%%%%%%%%%%%%
%   This discrete optimization is solved by Branch and Bound, which is a method
%   to minimize a function over a set. It requires us to (1) efficiently specify
%   solution subsets, (2) determine a lower bound on the minimal cost of the
%   solutions within a subset, and (3) specify a strategy to split a solution
%   subset into two new subsets.
% %\begin{enumerate}[itemsep=2pt,parsep=0pt]
% %  \item Start with the set of all elements $\SET Q=\{ \AllSelections \}$
% %  \item \textbf{Repeat}:
% %  \begin{enumerate}[itemsep=2pt,parsep=0pt,topsep=2pt]
% %  \item  Take the minimal subset\vspace{-0.7em}
% %    \begin{align}
% %      \J_i &\leftarrow \argmin_{\J_i \in \SET Q} g(\J_i)\\
% %                  \SET Q   &\leftarrow \SET Q \setminus \{ \J_i \}\nonumber
% %    \end{align}
% %  \item
% %    \textbf{Return} $\Selection$ \textbf{if} $\J_i=\{\Selection\}$ is a single element.
% %  \item Split $\J_i$ into \vspace{-0.7em}
% %    \begin{align}
% %      \J_i^1 &\subset \J_i, \J_i^2 \subset \J_i\\\quad\text{ s.t. }\J_i &= \J_i^1 \cup \J_i^2.\nonumber
% %    \end{align}
% %  \item Add the new subsets to the candidates\vspace{-0.7em}
% %    \begin{align}
% %      \SET Q &\leftarrow \SET Q \cup \{ \J_i^1, \J_i^2 \}\nonumber
% %    \end{align}
% %  \end{enumerate}
% %\end{enumerate}
%
%   The ingredients in our case are:
%  \begin{enumerate}
%  \item Solution subsets are created by taking subsets of landmark
%  candidates, and considering the Kartesian product of all selected landmark
%  candidates
%  \item We bound the cost for such a solution set by taking for each
%  landmark the minimal distance to the convex hull of the selected candidates
% % \begin{align}
% %   g(\J) &= \min_{\Params} \sum_i \rho\left( d_{\text{convex hull}}(\l_i^{\J_i}, m_i(\Params)) \right)\\
% %         &< \min_{\Params} \min %\nonumber\\
% %   d_{\text{convex hull}}(\l_i^\J, \VEC x) &=  \min_{c \in \text{convex hull}(\l_i^{\J})}\normLR{ x - c }.
% % \end{align}
%  \item We found that splitting landmark candidates such that the convex hull of the resulting two landmark candidates are as distant as possible is most effective.
%  \end{enumerate}
% \end{posterbox}

 \end{poster}
\end{document}

